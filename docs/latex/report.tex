\documentclass[12pt,a4paper]{article}

% Packages
\usepackage[utf8]{inputenc}
\usepackage[T1]{fontenc}
\usepackage[english]{babel}
\usepackage{amsmath,amsfonts,amssymb,amsthm}
\usepackage{graphicx}
\usepackage{hyperref}
\usepackage{listings}
\usepackage{xcolor}
\usepackage{algorithm}
\usepackage{algpseudocode}
\usepackage{geometry}
\usepackage{setspace}
\usepackage{float}
\usepackage{caption}
\usepackage{subcaption}

% Page setup
\geometry{
    a4paper,
    left=3cm,
    right=2cm,
    top=3cm,
    bottom=2cm
}

\onehalfspacing

% Hyperref setup
\hypersetup{
    colorlinks=true,
    linkcolor=blue,
    filecolor=magenta,      
    urlcolor=cyan,
    pdftitle={Graph Algorithms with GRASP - Technical Report},
    pdfauthor={Fabricio Guidine}
}

% Algorithm setup
\floatname{algorithm}{Algorithm}
\renewcommand{\algorithmicrequire}{\textbf{Input:}}
\renewcommand{\algorithmicensure}{\textbf{Output:}}

% Code listing setup
\lstset{
    language=C++,
    basicstyle=\ttfamily\small,
    keywordstyle=\color{blue}\bfseries,
    commentstyle=\color{green!60!black},
    stringstyle=\color{red},
    numbers=left,
    numberstyle=\tiny\color{gray},
    stepnumber=1,
    numbersep=5pt,
    frame=single,
    breaklines=true,
    showstringspaces=false,
    tabsize=2
}

% Title information
\title{Graph Algorithms with GRASP\\
       \large Implementation and Analysis of Fundamental Graph Algorithms}
\author{Fabricio Guidine}
\date{\today}

\begin{document}

% Cover page
\vspace*{-2cm}
{\bf
\begin{center}
{\large
\hspace*{0cm}Federal University of Juiz de Fora} \\
\hspace*{0cm}Department of Computer Science \\
\hspace*{0cm}Graph Theory \\
\end{center}}
\vspace{3.0cm}
\noindent
\begin{center}
{\Large \bf Graph Algorithms with GRASP} \\[0.5cm]
{\large Implementation and Analysis of Fundamental Graph Algorithms} \\[3cm]
{\Large \textbf{Group:}}\\[10mm]
{ Fabricio Guidine }\\[3mm]
{ Walkíria Garcia }\\[15mm]

{\Large Professor: Stênio Sã Rosário F. Soares}\\[1.0cm]
\end{center}

{\raggedleft
\begin{minipage}[t]{6.3cm}
\setlength{\baselineskip}{0.25in}
Technical report of the final project for the course DCC059 - Graph Theory, part of the course evaluation.
\end{minipage}\\[1cm]}

\vspace{0.2cm}
{\center Juiz de Fora \\[3mm]
November 2019 \\}

\newpage


% Table of contents
\newpage
\tableofcontents
\newpage

% List of figures
\listoffigures
\newpage

% List of tables
\listoftables
\newpage

% Main content
\pagenumbering{arabic}
\setcounter{page}{1}

\section{Introdução}

Este relatório apresenta a implementação e análise de algoritmos fundamentais de grafos, incluindo algoritmos de busca, caminhos mínimos, árvores geradoras mínimas e metaheurísticas. O trabalho foi desenvolvido como parte da disciplina de Teoria dos Grafos, utilizando a linguagem C++ e seguindo uma arquitetura modular e organizada.

O projeto implementa uma biblioteca completa de algoritmos de grafos, incluindo:

\begin{itemize}
    \item Estruturas de dados fundamentais (grafo, nó, aresta)
    \item Algoritmos de busca (BFS, DFS)
    \item Algoritmos de caminhos mínimos (Dijkstra, Floyd-Warshall)
    \item Algoritmos de árvores geradoras mínimas (Prim, Kruskal)
    \item Componentes fortemente conexas (Kosaraju)
    \item Ordenação topológica (Kahn)
    \item Metaheurística GRASP (Greedy Randomized Adaptive Search Procedure)
\end{itemize}

\section{Revisão Bibliográfica}

\subsection{Estruturas de Dados para Grafos}

Grafos podem ser representados de diversas formas, sendo as mais comuns a lista de adjacências e a matriz de adjacências. A representação por lista de adjacências é mais eficiente em termos de espaço para grafos esparsos, enquanto a matriz de adjacências é preferível para grafos densos \cite{cormen2009}.

\subsection{Algoritmos de Busca}

Os algoritmos de busca em largura (BFS) e profundidade (DFS) são fundamentais para a exploração de grafos. O BFS utiliza uma fila para explorar os vértices nível por nível, enquanto o DFS utiliza uma pilha (ou recursão) para explorar o máximo possível antes de retroceder \cite{cormen2009}.

\subsection{Algoritmos de Caminhos Mínimos}

O algoritmo de Dijkstra encontra os caminhos mínimos de uma origem para todos os outros vértices em grafos com pesos não-negativos, com complexidade $O((V + E) \log V)$ usando heap de Fibonacci \cite{dijkstra1959}.

O algoritmo de Floyd-Warshall resolve o problema de caminhos mínimos entre todos os pares de vértices, com complexidade $O(V^3)$ \cite{floyd1962}.

\subsection{Árvores Geradoras Mínimas}

Os algoritmos de Prim e Kruskal são utilizados para encontrar árvores geradoras mínimas em grafos não direcionados e conectados. Prim utiliza uma abordagem gulosa baseada em vértices, enquanto Kruskal utiliza uma abordagem baseada em arestas com estrutura de dados union-find \cite{cormen2009}.

\subsection{Metaheurísticas}

O GRASP (Greedy Randomized Adaptive Search Procedure) é uma metaheurística que combina construção gulosa randomizada com busca local. É amplamente utilizado para problemas de otimização combinatória \cite{resende2016}.

\section{Metodologia}

\subsection{Estrutura do Projeto}

O projeto foi organizado em uma arquitetura modular:

\begin{itemize}
    \item \texttt{include/}: Arquivos de cabeçalho (.h)
    \item \texttt{src/core/}: Implementações das estruturas de dados fundamentais
    \item \texttt{src/algorithms/}: Implementações dos algoritmos
    \item \texttt{src/utils/}: Utilitários e funções auxiliares
    \item \texttt{tests/}: Testes unitários usando Google Test
\end{itemize}

\subsection{Implementação}

Todas as implementações foram realizadas em C++17, utilizando:
\begin{itemize}
    \item Estruturas de dados orientadas a objetos
    \item Namespaces para organização do código
    \item Templates e STL quando apropriado
    \item Gerenciamento manual de memória com ponteiros
\end{itemize}

\subsection{Testes}

Os testes foram implementados utilizando o framework Google Test, cobrindo:
\begin{itemize}
    \item Operações básicas de grafos
    \item Correção dos algoritmos
    \item Casos extremos e limites
\end{itemize}

\section{Implementação}

\subsection{Estruturas de Dados}

\subsubsection{Classe Edge}

A classe \texttt{Edge} representa uma aresta do grafo, contendo:
\begin{itemize}
    \item ID do vértice destino
    \item Peso da aresta
    \item Ponteiro para a próxima aresta
\end{itemize}

\subsubsection{Classe Node}

A classe \texttt{Node} representa um vértice do grafo, contendo:
\begin{itemize}
    \item ID do vértice
    \item Lista de arestas adjacentes
    \item Grau de entrada e saída
    \item Peso do vértice (se aplicável)
\end{itemize}

\subsubsection{Classe Graph}

A classe \texttt{Graph} é a estrutura principal, suportando:
\begin{itemize}
    \item Grafos direcionados e não direcionados
    \item Grafos ponderados e não ponderados
    \item Operações de inserção e remoção de vértices e arestas
    \item Análise de conectividade e circuitos
\end{itemize}

\subsection{Algoritmos Implementados}

\subsubsection{Busca em Largura (BFS)}

O algoritmo BFS foi implementado utilizando uma fila (\texttt{std::queue}) para garantir a exploração nível por nível.

\begin{algorithm}[H]
\caption{Busca em Largura}
\begin{algorithmic}[1]
\Require Grafo $G$, vértice inicial $s$
\Ensure Ordem de visita dos vértices
\State Criar fila $Q$
\State Marcar $s$ como visitado
\State $Q$.enqueue($s$)
\While{$Q$ não está vazia}
    \State $v \gets Q$.dequeue()
    \State Processar $v$
    \For{cada vértice $u$ adjacente a $v$}
        \If{$u$ não foi visitado}
            \State Marcar $u$ como visitado
            \State $Q$.enqueue($u$)
        \EndIf
    \EndFor
\EndWhile
\end{algorithmic}
\end{algorithm}

\subsubsection{Algoritmo de Dijkstra}

Implementado utilizando uma fila de prioridade (\texttt{std::priority\_queue}) para eficiência.

\begin{algorithm}[H]
\caption{Algoritmo de Dijkstra}
\begin{algorithmic}[1]
\Require Grafo ponderado $G$, vértice origem $s$
\Ensure Distâncias mínimas de $s$ para todos os vértices
\State Inicializar distâncias $d[v] \gets \infty$ para todo $v$
\State $d[s] \gets 0$
\State Criar fila de prioridade $Q$ com $(0, s)$
\While{$Q$ não está vazia}
    \State $(d_u, u) \gets Q$.extractMin()
    \If{$u$ já foi processado} \textbf{continue} \EndIf
    \State Marcar $u$ como processado
    \For{cada aresta $(u, v)$ com peso $w$}
        \If{$d[u] + w < d[v]$}
            \State $d[v] \gets d[u] + w$
            \State $Q$.insert($(d[v], v)$)
        \EndIf
    \EndFor
\EndWhile
\end{algorithmic}
\end{algorithm}

\subsubsection{Algoritmo de Floyd-Warshall}

Implementado com três loops aninhados para calcular todas as distâncias entre pares.

\subsubsection{Algoritmo de Prim}

Utiliza uma fila de prioridade para selecionar iterativamente a aresta de menor peso que conecta um vértice na MST a um vértice fora dela.

\subsubsection{Algoritmo de Kruskal}

Utiliza a estrutura de dados union-find para verificar se a adição de uma aresta cria um ciclo.

\subsubsection{Componentes Fortemente Conexas (Kosaraju)}

Implementa o algoritmo de Kosaraju em duas fases:
\begin{enumerate}
    \item Primeira DFS no grafo original para determinar ordem de término
    \item Segunda DFS no grafo transposto na ordem inversa
\end{enumerate}

\subsubsection{Ordenação Topológica (Kahn)}

Utiliza o algoritmo de Kahn baseado em graus de entrada.

\subsubsection{GRASP}

Implementa a metaheurística GRASP com:
\begin{itemize}
    \item Construção gulosa randomizada com lista de candidatos restrita (RCL)
    \item Busca local para melhorar soluções
    \item Múltiplas iterações para encontrar melhor solução
\end{itemize}

\section{Resultados}

\subsection{Testes Unitários}

Todos os testes unitários foram executados com sucesso, validando:
\begin{itemize}
    \item Operações básicas de inserção e remoção
    \item Correção dos algoritmos de busca
    \item Correção dos algoritmos de caminhos mínimos
    \item Correção dos algoritmos de MST
    \item Detecção de componentes fortemente conexas
    \item Ordenação topológica
\end{itemize}

\subsection{Análise de Complexidade}

A Tabela~\ref{tab:complexity} apresenta a complexidade temporal e espacial dos algoritmos implementados.

\begin{table}[H]
\centering
\caption{Complexidade dos Algoritmos}
\label{tab:complexity}
\begin{tabular}{|l|c|c|}
\hline
\textbf{Algoritmo} & \textbf{Tempo} & \textbf{Espaço} \\
\hline
BFS/DFS & $O(V + E)$ & $O(V)$ \\
Dijkstra & $O((V + E) \log V)$ & $O(V)$ \\
Floyd-Warshall & $O(V^3)$ & $O(V^2)$ \\
Prim & $O((V + E) \log V)$ & $O(V)$ \\
Kruskal & $O(E \log E)$ & $O(V)$ \\
Topological Sort & $O(V + E)$ & $O(V)$ \\
SCC (Kosaraju) & $O(V + E)$ & $O(V)$ \\
\hline
\end{tabular}
\end{table}

\section{Conclusão}

Este trabalho apresentou uma implementação completa e organizada de algoritmos fundamentais de grafos em C++. A arquitetura modular permite fácil manutenção e extensão, enquanto os testes unitários garantem a correção das implementações.

Os algoritmos implementados cobrem as principais classes de problemas em grafos: busca, caminhos mínimos, árvores geradoras mínimas, análise de conectividade e otimização combinatória através de metaheurísticas.

Futuras melhorias podem incluir:
\begin{itemize}
    \item Otimizações adicionais para grafos muito grandes
    \item Implementação de mais metaheurísticas
    \item Análise experimental com instâncias reais
    \item Visualização gráfica dos resultados
\end{itemize}

\section{Referências}

\begin{thebibliography}{99}

\bibitem{cormen2009}
Cormen, T. H., Leiserson, C. E., Rivest, R. L., \& Stein, C. (2009). 
\textit{Introduction to Algorithms} (3rd ed.). MIT Press.

\bibitem{dijkstra1959}
Dijkstra, E. W. (1959). 
A note on two problems in connexion with graphs. 
\textit{Numerische Mathematik}, 1(1), 269-271.

\bibitem{floyd1962}
Floyd, R. W. (1962). 
Algorithm 97: Shortest path. 
\textit{Communications of the ACM}, 5(6), 345.

\bibitem{resende2016}
Resende, M. G., \& Ribeiro, C. C. (2016). 
\textit{Optimization by GRASP: Greedy Randomized Adaptive Search Procedures}. 
Springer.

\end{thebibliography}

\end{document}

